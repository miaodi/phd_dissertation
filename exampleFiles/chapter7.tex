\chapter{Conclusions and future work}
\label{chp:chapter7}
\graphicspath{{figures/}{figures/chapter6/}}
\pgfplotsset{
  table/search path={{figures/chapter6/data},{data}},
}

In this dissertation, an isogeometric analysis based patch coupling framework for higher order PDEs is developed. Mathematical analysis and numerical examples veriefy the accuracy and robustness of this technology. The use of dual basis with compact support significantly increases the sparsity of the constrained linear system and reduces the computational cost. In addition, this technology provides a unified formulation when dealing with vertices with different valences, which makes it an ideal analysis technology for higher order problems over unstructured meshes. The main contributions can be summarized as:
\begin{itemize}
  \item Formulation of abstract dual mortar method for both homogeneous and non-homogeneous constraints.
  \item Development of \Bezier dual mortaring framework for higher order problems, including biharmonic problem, phase-field problem and Kirchhoff-Love shell problem.
  \item Development and implementation of different vertex treatments for multi-patch coupling problem.
  \item Development of the enriched \Bezier dual basis, which improves the approximation ability of the \Bezier dual basis.
  \item Development and implementation of two locking-free formulations for both Timoshenko beam and linear elasticity problem. 
  \item Implementation of a C++ based multi-thread isogeometric analysis code which is utilized in:
  \begin{itemize}
    \item Simulations of Poisson and biharmonic problems
    \item Simulations of wave propagation problems
    \item Simulations of coupling problems based on the discontinuous Galerkin method
    \item Simulations of linear elasticity and Timoshenko beam problems
    \item Simulations of phase field problems
    \item Simulations of Kirchhoff-Love shell problems
  \end{itemize}
\end{itemize}

There exist many potential future work we can perspect from this work. From the modelling aspect, a potential extension of the present work is to incorperate the constrained NURBS patches directly into the CAD process. The dual mortar formulation provides a direct access to locally supported basis functions in the constrained space, however, these basis functions are not guaranteed to be positive over their supports and form a partition of unity, which are of crucial importance for the CAD community. Hence, in order to use the constrained basis functions as the design space, a better formulation is needed to accommodate these two properties. The support size of the constrained basis function is directly linked to the support size of dual basis functions. However, current enrichment procedure improves the polynomial reproduction at the expense of support size of enriched \Bezier dual basis functions. We believe there exists a better formulation that can achieve the same performance without any influences on the support size.\par

There are many potential research topics in the analysis aspect as well. Since the focus of this research is on the development of a coupling formulation, we assume all materials tested in this work to be linear elastic. Hence, the verification of the dual mortar formulation over non-linear material is needed. The vibration example proved that the weak-$C^1$ coupling scheme can significantly reduce the highest eigenvalue. Hence, it is interested to see the performance of this work in explicit dynamics. $C^2$ continuous functions are required for solving $6^\text{th}$ order PDEs, including the triharmonic equation~\cite{tagliabue2014isogeometric, bartezzaghi2015isogeometric}, the phase-field crystal equation~\cite{gomez2012unconditionally} and Kirchhoff-Love shell with strain gradient elasticity~\cite{balobanov2019kirchhoff}. Fortunately, it seems that most of the theory and formulation developed in this dissertation can be directly applied in the formulation of weak-$C^2$ continuity. For the coupling of Kirchhoff-Love shell, the dual mortar compatible constraint in this work is slightly stronger than what is required by the problem and may lead to sub-optimal convergence for patches connected at a kink. Hence, the development of a better constraint for Kirchhoff-Love shell is still needed. In addition, shell structures are often connected with solid components (e.g. stiffner), an extension to handle solid-shell coupling is of crucial importance. Furthermore, in the development of dual basis based locking-free element, our study is restricted to the small deformation scenario, however, an extension to the large deformation scenario is of crucial importance. In all of these directions, the use of isogeometric analysis should lead to improvements in both efficiency and accuracy. As a result, these direction may become potentially fruitful research areas in the future.