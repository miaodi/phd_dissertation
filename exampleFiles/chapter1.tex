\chapter{Introduction}
\label{chp:chapter1}
\graphicspath{{figures/}{figures/chapter1/}}
\pgfplotsset{
    table/search path={{figures/chapter1/data},{data}},
}

Isogeometric analysis, introduced by Hughes et al.~\cite{HUGHES20054135}, leverages computer aided design (CAD) representations directly in finite element analysis. It has been shown that this approach can alleviate the model preparation burden of going from a CAD design to an analysis model and improve overall solution accuracy and robustness~\cite{bazilevs2006isogeometric, da2011some, da2014mathematical}. Additionally, the higher-order smoothness inherent in CAD basis functions make it possible to solve higher-order partial differential equations, e.g. the biharmonic equation~\cite{kapl_isogeometric_2015, kapl_isogeometric_2017}, the Kirchhoff-Love shell problem~\cite{kiendl2009isogeometric, kiendl2010bending, kiendl2015isogeometric} and the Cahn-Hilliard equation~\cite{gomez2008isogeometric, borden2014higher} directly without resorting to complex mixed discretization schemes.\par

CAD models are often built from collections of non-uniform rational B-splines (NURBS). Adjacent NURBS patches often have inconsistent knot layouts, different parameterizations, and may not even be physically connected. Additionally, trimming curves~\cite{kim2009isogeometric, schmidt2012isogeometric} are often employed to further simplify the design process and to extend the range of objects that can be modeled by NURBS at the expense of further complicating the underlying parameterization of the object. While usually not an issue from a design perspective, these inconsistencies in the NURBS patch layout, including trimming, must be accommodated in the isogeometric model to achieve accurate simulation results. As shown in Figure~\ref{fig:geometries}, two primary approaches are often employed. First, the exact trimmed CAD model, shown in Figure~\ref{fig:geometries} in the middle, is used directly in the simulation~\cite{schmidt2012isogeometric}. To accomplish this requires additional algorithms for handling cut cells and the weak imposition of boundary conditions and may result in reduced solution accuracy and robustness. Second, the CAD model is reparameterized~\cite{xu2014high}, as shown in Figure~\ref{fig:geometries} on the right, into a watertight spline representation like multi-patch NURBS, subdivision surfaces~\cite{peters2008subdivision}, or T-splines~\cite{sederberg_t-splines_2003} which can then be used as a basis for analysis directly. The reparameterization process often results in more accurate and robust simulation results but is only semi-automatic using prevailing approaches. In both cases, existing techniques are primarily surface-based due to the predominance of surface-based CAD descriptions.

\begin{figure}[ht]
	\captionsetup[subfigure]{labelformat=empty, font = footnotesize}
	\centering
	\begin{subfigure}[b]{0.32\textwidth}
		\centering
		\includestandalone[scale=.7]{geometry}
		\caption{A geometry}
	\end{subfigure}
	\begin{subfigure}[b]{0.32\textwidth}
		\centering
		\includestandalone[scale=.7]{trimmed_geometry}
		\caption{Trimmed model}
	\end{subfigure}
	\begin{subfigure}[b]{0.32\textwidth}
		\centering
		\includestandalone[scale=.7]{reparameterized_geometry}
		\caption{Reparameterized model}
	\end{subfigure}
	\caption{A geometry and two modelling strategies: trimming and reparameterization.}
	\label{fig:geometries}
\end{figure}

From the analysis perspective, the main challenge for conducting finite element analysis over a geometry consisting of multiple spline patches is how to efficiently and accurately exchange information among different patches. In this dissertation, we focus on the \textit{dual mortar method}, which can robustly apply constraints over intersections of reparameterized multi-patch geometries.  

\section{State of the art}