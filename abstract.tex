% This is just to make sure that if it goes onto multiple pages that
% it will only be on odd pages.
\afterpage{\cleardoublepage}
% Some people have had problems with needing a little more space or a little less space right before the body of the abstract. If you have that problem, you can uncomment the next line, and either add or take away space manually (negative spaces are OK).
%\vspace*{-0.05in}

Isogeometric analysis is aimed to mitigate the gap between Computer-Aided Design (CAD) and analysis by using a unified geometric representation. Thanks to the exact geometry representation and high smoothness of adopted basis functions, isogeometric analysis demonstrated excellent mathematical properties and successfully addressed a variety of problems. In particular, it allows to solve higher order Partial Differential Equations (PDEs) directly omitting the usage of mixed approaches. Unfortunately, complex CAD geometries are often constituted by multiple Non-Uniform Rational B-Splines (NURBS) patches and cannot be directly applied for finite element analysis.\par

In this work, we presents a dual mortaring framework to couple adjacent patches for higher order PDEs. The development of this formulation is initiated over the simplest $4^\text{th}$ order problem--biharmonic problem. In order to speed up the construction and preserve the sparsity of the coupled problem, we derive a dual mortar compatible $C^1$ constraint and utilize the \Bezier dual basis to discretize the Lagrange multipler spaces. We prove that this approach leads to a well-posed discrete problem and specify requirements to achieve optimal convergence. \par

After identifying the cause of sub-optimality of \Bezier dual basis, we develop an enrichment procedure to endow \Bezier dual basis with adequate polynomial reproduction ability. The enrichment process is quadrature-free and independent of the mesh size. Hence, there is no need to take care of the conditioning. In addition, the built-in vertex modification yields compatible basis functions for multi-patch coupling.\par

To extend the dual mortar approach to couple Kirchhoff-Love shell, we develop a dual mortar compatible constraint for Kirchhoff-Love shell based on the Rodrigues' rotation formula. This constraint provides a unified formulation for both smooth couplings and kinks. The enriched \Bezier dual basis preserves the sparsity of the coupled Kirchhoff-Love shell formulation and yields accurate results for several benchmark problems.\par

Like the dual mortaring formulation, locking problem can also be derived from the mixed formulation. Hence, we explore the potential of \Bezier dual basis in alleviating transverse shear locking in Timoshenko beams and volumetric locking in nearly compressible linear elasticity. Interpreting the well-known $\bar{B}$ projection in two different ways we develop two formulations for locking problems in beams and nearly incompressible elastic solids.  One formulation leads to a sparse symmetric symmetric system and the other leads to a sparse non-symmetric system. 

